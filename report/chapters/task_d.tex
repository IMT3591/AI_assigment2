
\chapter{Task D}
\section{Dynamic environment}

\subsection{Full sensing}
With full sensing the agent can move continously and recalculate the optimal
path when an environment change occurs. Since fully senses the environment it
will take any obstructions into account when calculating the optimal path and
next move. Therefore it will not hit any objects and continously move around
paths that are obstructed.


\subsection{Partial Sensing}
When the agent senses or encounters an obstruction it must stop and retreat to
the previous Vertice or stay on the current Vertice. Then it has to update the 
environment and re-calculate the optimal path to decide which move to make next.

\subsection{No waypoint addition}
The inability to create new Vertices and Edges will hinder the possibility of
discovering a shortcut to a Vertice or an intermediate vertice between two
vertices.  This might result in a complete halt in the graph traversal or
inability to find a result at all.

If e.g one has a map of an area and it contains an articulation point. If this
articulation point is obstructed one is not able to traverse the graph. If one
has the ability to discover and add new edges and vertices one might find an
edge or vertice that connects the two sub graphs and avoids the articulation
point all together.

Another reason why it would cause a problem is if a NPC in a game has a blocked
path in the environment which is later unblocked. Inability to update this and
add a new vertice and edge to the environment will make it unable to traverse
this way, in which case it would come to a halt and gameplay would be very poor.

In the following case it would make sense to have the possibility of adding
additional edges and vertices.  If however the task is to find the optimal path
in the current environment it wouldn't make sense to have this ability.





