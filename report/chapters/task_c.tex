
\chapter{Task C}
This is the explanation of the different terms listed below, our results for
this is listed below in their respective sections.
\begin{description}
\item[Completeness:			]
	If there exists a goal state, will the agent find it during runtime?
\item[Optimality:				]
	Wheter the solution the agent finds is in fact the optimal path from A to B.
\item[Time Complexity:	]
	How long it will take for the agent to find the goal, if any.  We are
	measuring this in the following manner -- \textit{How many lines of codes is
	being executed during runtime}.  This is because it will yield a common value
	not influenced by CPU-time and inaccurate clock measurments.
\item[Memory Complexity:]
	How much memory is required by the program to find the solution. We measure
	this using Valgrind to compute how much memory is being used by the program.
\end{description}

\section{A* Tree Search}
\begin{description}
\item[Completeness:			]
\item[Optimality:				]
\item[Time Complexity:	]
\item[Memory Complexity:]
\end{description}

\begin{table}[h]
\centering
\begin{tabular}{	p{0.1\textwidth} p{0.1\textwidth} 
									p{0.2\textwidth} p{0.2\textwidth} }\hline
	Start & End & Time & Tot Memory \\\hline
	1		&	24 	& 	&	\\
	24	&	1 	& 	&	\\
	10	&	5 	& 	&	\\
	5		&	10 	& 	&	\\
	1		&	20 	& 	&	\\
	20	&	1		&		&	\\
\end{tabular}
\caption{Measured values for A* Tree Search}\label{tbl:sumTree}
\end{table}

\section{A* Graph Search}
\begin{description}
\item[Completeness:			]
\item[Optimality:				]
\item[Time Complexity:	]
\item[Memory Complexity:]
\end{description}

\begin{table}[h]
\centering
\begin{tabular}{	p{0.1\textwidth} p{0.1\textwidth} 
									p{0.2\textwidth} p{0.2\textwidth} }\hline
	Start & End & Time & Tot Memory \\\hline
	1		&	24 	& 	&	\\
	24	&	1 	& 	&	\\
	10	&	5 	& 	&	\\
	5		&	10 	& 	&	\\
	1		&	20 	& 	&	\\
	20	&	1		&		&	\\
\end{tabular}
\caption{Measured values for A* Graph Search}\label{tbl:sumGraph}
\end{table}




